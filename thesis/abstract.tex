\newenvironment{abstractpage}
  {\cleardoublepage\thispagestyle{empty}}
  {\vfill\cleardoublepage}
\newenvironment{abstract}[1]
  {\bigskip
   \begin{center}\bfseries#1\end{center}\small\leftskip=0.5cm\rightskip=0.5cm}
  {\par\bigskip}

\providecommand{\keywords}[2]{\footnotesize\textbf{\textit{#1:}} #2}

\begin{abstractpage}
\begin{abstract}{Abstract}

Most network defense systems only rely on evidence-based knowledge about past cyberattacks, known as threat intelligence. Firewalls and intrusion prevention systems rely on the shared threat intelligence generated by other systems to prevent the attacks even before they happen.
Such intelligence is usually shared via centralized public blocklists or databases where a single centralized authority has complete control over what will be published and what will be censored. In addition to that, it is a central point of failure.

To mitigate a central point of failure, one can use peer-to-peer networks to share the threat intelligence. However, because these networks are open to anyone, including malicious actors, the peers need to be able to determine who to trust and which data is better to discard.

In this thesis, we introduce Fides. Fides is a generic trust model fine-tuned for sharing threat intelligence in highly adversarial global peer-to-peer networks of intrusion prevention agents.
We build Fides with the lessons learned from other state-of-the-art trust models and optimized it for the broad spectrum of environments in the peer-to-peer network where anybody can join and leave at any time.
Apart from uncovering the behavior of any generic peer in the network, Fides is also able to identify peer's membership in organizations and utilize this knowledge for trust management.
Moreover, Fides allows administrators to filter out private data and share them only inside a single organization or with the most trusted peers. 
At the end of our work, we prove that Fides is a suitable solution for modeling trust in the global peer-to-peer networks. We show that when Fides talks to at least 25\% of peers that are part of trusted organizations, the rest of the network can be controlled by the malicious actors, and Fides is still able to determine correct threat intelligence.

\end{abstract}

\keywords{Keywords}{trust models, threat intelligence, collaborative network defense, intelligence sharing}

\vspace*{\fill}

\newpage
\begin{abstract}{Abstrakt}
    TBD \todo{add cz absrakt}
    
\end{abstract}
\keywords{Klíčová slova}{TBD1, TBD2} \todo{add cz kws}

\end{abstractpage}
\thispagestyle{empty}

\cleardoublepage
\newenvironment{abstractpage}
  {\cleardoublepage\thispagestyle{empty}}
  {\vfill\cleardoublepage}
\newenvironment{abstract}[1]
  {\bigskip
   \begin{center}\bfseries#1\end{center}\small\leftskip=0.5cm\rightskip=0.5cm}
  {\par\bigskip}

\providecommand{\keywords}[2]{\footnotesize\textbf{\textit{#1:}} #2}

\begin{abstractpage}
\begin{abstract}{Abstract}

Most network defense systems only rely on evidence-based knowledge about past cyberattacks, known as threat intelligence. Firewalls and intrusion prevention systems rely on the shared threat intelligence generated by other systems to prevent attacks before is too late.
Such threat intelligence is usually shared via centralized public and private blocklists where a single centralized authority, hopefully, has complete control over what is published. Such centralized systems has many issues: single point of failure both technically and in trust, lack of flexibility on new data and providers, and manual trust in the providers.

To mitigate these problems, peer-to-peer networks can be used to share threat intelligence. However, because these networks are open to anyone, including malicious actors, the peers need to be able to determine who to trust and which data is better to discard.

This thesis introduces Fides. Fides is a generic trust model fine-tuned for sharing security threat intelligence in highly adversarial global peer-to-peer networks of intrusion prevention agents.
We design and built Fides taking into account the problems and limitations of previous state-of-the-art trust models, optimizing it for a broad spectrum of peer-to-peer networks where peers can join and leave at any time.
Fides evaluates the behavior of peers in the network, including their membership in pre-trusted organizations and uses this knowledge to compute the trust.
Fides continually assesses received data from the peers and by weighting and comparing them with each other as well as with the existing knowledge Fides is able to determine which peer provides better threat intelligence and which peers are more reliable. The received threat intelligence is always aggregated and weighted and then provided to the underlying intrusion prevention system.
Among many results, our experiments show that if at least 25\% of peers are part of trusted organizations, then 75\% of the network can be \textit{completely controlled by the malicious actors}, and Fides would still be able to provide the correct values of the threat intelligence data to the peers. We conclude that Fides is a suitable solution for modeling trust in global adversarial peer-to-peer networks. 

The direct contribution of this thesis is the computational model of the trust model Fides, the reference implementation of the model in Python, the simulation framework for modeling peers' behavior in the network including the implementation of the framework and the implementation of the Fides module for reference intrusion prevention system.

\end{abstract}

\keywords{Keywords}{trust models, threat intelligence, collaborative network defense, intelligence sharing}

\vspace*{\fill}

\newpage
\begin{abstract}{Abstrakt}
    TBD \todo{add cz absrakt}
    
\end{abstract}
\keywords{Klíčová slova}{TBD1, TBD2} \todo{add cz kws}

\end{abstractpage}
\thispagestyle{empty}

\cleardoublepage
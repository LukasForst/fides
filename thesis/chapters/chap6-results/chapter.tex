\chapter{Results}
\label{ch:results}

In this chapter, we evaluate the results of the simulations that were designed in the previous chapter \ref{ch:experiments}.

There are too many different situations, to evaluate each setup completely. Moreover, each situation can be modeled in the developed simulation framework that is part of the published source code~\cite{fidesGithub} and thus anybody can verify and simulate their own situation they are interested in.

For that reason, we focus mainly on evaluating Fides under specific conditions that verify its resilience in the situations with many byzantine peers.
We try to discover a situation where there are as much adversarial peers as possible and Fides is still able to guarantee that it is able to come up with correct target score.

Note, that all all figures in this chapter can be replicated by re-running the simulations that are stored in the package with the implementation $simulations/cases/figures$~\cite{fidesGithub}.
The graphs might be slightly different because the threat intelligence and recommendations are sampled as described in \ref{sec:sampling-threat-intelligence}, but the overall results should be the same.

\section{General Overview of a Single Simulation}
\label{sec:general-overview-of-simulation-output}
First, let us describe the outcome of the simulation framework which is visualized in the Figure~\ref{fig:single-simulation-example}.
The simulation framework provides this graph for each possible simulation.

\begin{figure}
    \centering
    \includegraphics[width=1.0\textwidth]{assets/example_evaluation.png}
    \caption{Example of a single simulation}
    \label{fig:single-simulation-example}
\end{figure}

The graph's headline explains which setup parameters were used for the trust model. In the case of Figure~\ref{fig:single-simulation-example} Fides used interaction evaluation strategy $MaxConfidenceTIEvaluation$ (Section~\ref{subsec:MaxConfidenceTIEvaluation}).
For aggregating the threat intelligence, Fides used the aggregation described in  Section~\ref{sec:network-intelligence-aggregation}.
The local Slips instance behaved like confident correct peer outlined in  Section~\ref{subsubsec:confident-correct-peer}.

The graph on top in Figure~\ref{fig:single-simulation-example} shows the development of the \textit{service trust} $st_{i, j}$ (Section~\ref{subsec:service-trust}) on the vertical axis over \textit{time} on the horizontal axis. As mentioned in the Section~\ref{sec:environment-simulation}, the time is measured in \textit{clicks}.
One can see multiple peers that were involved in the simulation and their respective behavior. All possible behaviors are described in the Section~\ref{sec:peers-behavioral-patterns}.
There were four different peers that were communicating with the local instance of Fides, two of them were \textit{confident correct}, one was an \textit{uncertain peer} and the last one was a \textit{malicious peer}.

The dotted line indicates the time when the malicious peers start lying.
One can see that during this first period, when the malicious peers were not lying \textit{(before the line)}, they were gaining the service trust.
In the case of Figure~\ref{fig:single-simulation-example} this happened at click 25 when the malicious peers started lying.
After that, it is clear that they started to lose the service trust.

The second graph in Figure~\ref{fig:single-simulation-example} shows the \textit{target score} during the time (\textit{clicks}).
The target score $S^{k}_{T}$ (Section~\ref{sec:network-intelligence-aggregation}) is the part of the aggregated network threat intelligence, that was computed from the scores and confidences provided by each peer.
The score was calculated by Fides at click $k$ for target $T$.

The score graph contains two different targets, one that is according to the ground truth malicious and the second, that was benign.
We also included the moving average value (indicated as MM) within the window of 10 clicks to make the graph clear.

Finally, third graph, displays the aggregated confidence $C^{k}_{T}$ (Section~\ref{sec:network-intelligence-aggregation}) over time (\textit{clicks}).
The graph is similar to the score, we include raw values for each time window and target as well as the moving average within the window of 10 clicks.

In this example output graph, it can be seen that Fides was clearly able to identify that the malicious peer started to lie after click 25 because of the service trust $st^{k}_{i,j}$ for this peer that fell down almost instantly.
At the same time, we can see that on the score graph, the $S^{k}_{T}$ for both targets were skewed and started to get closer to $0$ because the malicious peer had already gained service trust and thus the threat intelligence provided by it had an impact on the final $S^{k}_{T}$.
However, after Fides figured out, that the peer is lying, it lowered its service trust for this peer and the score started to recover closer to the baseline.

\section{Evaluation of Fides Resilience}
\label{sec:fides-resilience}

To evaluate the resilience of Fides in different scenarios, we need to find the optimal configuration for the following parameters in Fides: interaction evaluation strategy (Section~\ref{sec:interaction-evaluation-strategies}), threat intelligence aggregation function (Section~\ref{sec:network-intelligence-aggregation}) and initial reputation (Section~\ref{subsubsec:computing-reputation}). Each combination of parameters is evaluated in its capacity to correctly classify targets in \textit{any} network topology.\footnote{Distribution of correct/uncertain/incorrect/malicious peers in the network.}

In this section, we are focusing on finding the best possible combination of parameters for the worst possible scenario. In other words, we want to identify a setup, where the Fides can guarantee that it is eventually going to provide the correct data and will classify the targets correctly even though the malicious actor controls most of the network.

We shows two specific scenarios - one with no pre-trusted peers and one where there are 25\% of peers part of some pre-trusted organization. Because even the scenario with only 25\% peers shows, that in some case is Fides able to defend itself against the rest of the network, we do not show scenarios with more pre-trusted peers, but we include them in the appendix (Figure~\ref{fig:performance-all-setups-50-pretrusted}).

\subsection{Scenario With No Pre-Trusted Peers}
\label{subsec:scenario-with-0-pretrusted-peers}

In this scenario, there are no pre-trusted peers nor organizations and Fides needs to determine trust in each peer by itself.
We simulated environments starting with the 75\% of confident correct peers~(behavior from Section~\ref{subsubsec:confident-correct-peer}) up to 75\% malicious peers~(behavior from Section~\ref{subsubsec:malicious-peer}) and used all possible setups.

\cleartoleftpage % check if this appears on the left side 
\subsubsection{Target Detection Performance}

The target detection performance $tdp$ (Section~\ref{subsec:target-detection-performance-metric}) is the most important metric because it evaluates how good is Fides in the target classification - if the Fides is able to correctly come up to a conclusion that \textit{evil.com} is malicious target and \textit{google.com} is the benign.

Figure~\ref{fig:25-target-detection} visualizes the target detection performance on three different graphs where each of the graphs is a single interaction evaluation strategy.
Each graph then displays dots with different colors. Each color a single threat intelligence aggregation method in combination with different initial reputation value.
A single dot in the graph is the value of $tdp$ and in a case when the $tdp \geq 1$, it means that Fides made on average the wrong decision about the targets and classified them with the wrong label.
In other words, if $tdp \geq 1$, Fides classified benign targets as malicious and the other way around.
We included the \textit{red line} that shows $tdp = 1$ so if a dot is above the \textit{red line}, the Fides made incorrect target classification.
For that reason, we optimize the \textit{dots} to be \textit{below} the \textit{red line} (classifications being correct).

The horizontal axis in each graph measures the environment hardness explained in Section~\ref{subsec:environment-hardness}. It is important to note, that hardness essentially expresses how many peers that can provide correct data are in the simulation. For example, if the hardness is $10$, $100\%$ of peers inside the simulation are providing correct data and behave like confident correct peers.
Thus the higher the value of hardness is, the easier it is for the Fides to do correct classification.

Specifically in Figure~\ref{fig:0-target-detection} we can see that that in the \textit{easy} environment, most of the \textit{dots} are below the red line until the hardness gets close to $3$.
The metrics perform more or less the same as they are able to stay bellow the red line until $eh = 3$. In that situation, the best performance and thus the lowest $tdp$ has $ThresholdTIEvaluation$ in combination with $WeighedDistanceToLocalTIEvaluation$ and initial reputation of $0.95$.

Interestingly, the $DistanceBasedTIEvaluation$ in combination with $0$ initial reputation and $AverageConfidenceTIAggregation$ for threat intelligence aggregation, shows the same target classification performance in each environment - $tdp = 0$. This suggests that the method was unable to determine any trust for any of the peers. This is then later confirmed by the Figure~\ref{fig:0-peer-trust}.

% TODO: move the legend to the left
\begin{figure}[hp]
    \centering
    \includegraphics[height=0.9\textheight]{assets/0_target_detection.png}
    \caption{Target detection performance (vertical axis) for three different interaction evaluation strategies in different environments (horizontal axis) with \textbf{no pre-trusted peers}.}
    \label{fig:0-target-detection}
\end{figure}

\cleartoleftpage
\subsubsection{Peers Behavior Detection Performance}
\begin{figure}[hp]
    \centering
    \includegraphics[width=1.0\textwidth]{assets/0_peer_trust.png}
    \caption{Behavior of peer's trust metrics in the different environments for different Fides's setups with \textbf{no pre-trusted peers}. On the left side peer's behavior detection performance, on the right side peer's average trust.}
    \label{fig:0-peer-trust}
\end{figure}

Figure~\ref{fig:0-peer-trust} displays two important metrics which are related to how much does Fides trust the peers in the network. First is peer's behavior detection performance metric $pbdp$ (Section~\ref{subsec:peers-behavior-detection-performance-metric}) and the second is the peer's average trust.

On the left side, one can see the peer's behavior detection performance metric that measures how good was Fides in estimating the peer's behavior. The lower value of $pbdp$ the better because the Fides's service trust for the peer was closer to the real value used in the simulation.

On the right side, we show the peer's average trust metric. That is an average trust of Fides for each peer. We include this metric in order to see how much trust was Fides able to obtain for the peers in the network.
It is important to note, that there is no \textit{correct} or \textit{desired} value of this metric, because for example in the environment where there are all peers confident correct, the peer's average trust should be high, but in the environment with all byzantine peers, this metric should be low because Fides should not trust incorrect and malicious peers.

As suggested in the previous section while measuring target detection performance, the right graph for $DistanceBasedTIEvaluation$ in combination with $AverageConfidenceTIAggregation$ shows, that this setup is unable to determine trust for the peers and has average peer's trust close to $0$. This means that the trust model will almost always aggregate threat intelligence to score $0$ with confidence $0$ making it, in a fact, useless. 

\cleartoleftpage
\subsection{Scenario With 25\% of Pre-Trusted Peers}
\label{subsec:scenario-with-25-pretrusted-peers}

In this scenario, Fides talks to 25\% of pre-trusted peers with. We simulated environments starting with the 75\% of confident correct peers~(behavior from Section~\ref{subsubsec:confident-correct-peer}) up to 75\% malicious peers~(behavior from Section~\ref{subsubsec:malicious-peer}) and used all possible setups.

\subsubsection{Target Detection Performance}
\label{subsubsec:target-detection-performance}
% TODO: move the legend to the left
\begin{figure}[hp]
    \centering
    \includegraphics[height=0.9\textheight]{assets/25_target_detection.png}
    \caption{Target detection performance (vertical axis) for three different interaction evaluation strategies in different environments (horizontal axis) with \textbf{25\% pre-trusted peers}.}
    \label{fig:25-target-detection}
\end{figure}

Specifically in Figure~\ref{fig:25-target-detection} one can see that until the environment hardness $eh \geq 3$, all strategies help Fides to classify targets correctly.
That changes after the environment hardness $eh \leq 2.5$ when the $ThresholdTIEvaluation$ as well as $MaxConfidenceTIEvaluation$ misclassify the targets and the $tdp \geq 1$. In that case, all threat intelligence aggregation methods are the same and all of them misclassify the targets no matter what initial reputation is used.
However, the $DistanceBasedTIEvaluation$ strategy in combination with $AverageConfidenceTIAggregation$ method is able to still classify the targets correctly and maintain the $tdp \leq 1$ even under toughest conditions where there are 75\% of adversarial peers in the simulation.

When Fides used in the similar situation the threat intelligence aggregation method $WeightedAverageConfidenceTIAggregation$, it misclassified the targets in one simulation in when the hardest environment.
Thus, this method does not provide a guarantee that Fides will end up with correct classifications for every target.

\cleartoleftpage
\subsubsection{Peers Behavior Detection Performance}

\begin{figure}[hp]
    \centering
    \includegraphics[width=1.0\textwidth]{assets/25_peer_trust.png}
    \caption{Behavior of peer's trust metrics in the different environments for different Fides's setups with \textbf{25\% pre-trusted peers}. On the left side peer's behavior detection performance, on the right side peer's average trust.}
    \label{fig:25-peer-trust}
\end{figure}

Figure~\ref{fig:25-peer-trust}, similarly to Figure~\ref{fig:0-peer-trust} shows the peer's behavior detection performance $pbdp$ on the left side and peer's average trust on the right side.
When we compare Figure~\ref{fig:0-peer-trust} (no pre-trusted peers) with this Figure~\ref{fig:25-peer-trust} (25\% pre-trusted peers), we can clearly see that especially the $pbdp$ metric improved and in all environment it holds that $pbdp \leq 0.4$. 
This means that Fides's ability to identify the true behavior of the peers greatly improved for all interaction evaluation strategies.

The biggest improvement was in strategy $DistanceBasedTIEvaluation$ which had poor performance with no pre-trusted peers in Figure~\ref{fig:0-peer-trust}. However, in the situation with 25\% pre-trusted peers it is now able to detect the true behavior of the peers with the similar precision as the other strategies.


\newpage
\section{Other Findings}
\label{sec:other-findings}

Even though the results from the previous Section~\ref{sec:fides-resilience} suggest that a combination of $DistanceBasedTIEvaluation$ for evaluating the interactions in combination with $AverageConfidenceTIAggregation$ is the best, this is not always true.

For example, recall Figure~\ref{fig:single-simulation-example} from the Section~\ref{sec:general-overview-of-simulation-output}, where the presented situation uses $MaxConfidenceTIEvaluation$ and it is able to correctly detect all types of peers as well as correctly determine the score for the target.
However, if we take the same environment and the only difference is using $DistanceBasedTIEvaluation$ for evaluating interactions, we get the following graph for the service trust in the Figure~\ref{fig:zero-gained-trust}.

\begin{figure}[ht]
    \centering
    \includegraphics[width=0.8\textwidth]{assets/zero_gained_trust.png}
    \caption{$DistanceBasedTIEvaluation$ in the situation from the figure~\ref{fig:single-simulation-example}}
    \label{fig:zero-gained-trust}
\end{figure}

The graph for confidence as well as target score for the situation from Figure~\ref{fig:zero-gained-trust} can be seen in the appendix in the Figure~\ref{fig:zero-gained-trust-all}.

The service trust graph in Figure~\ref{fig:zero-gained-trust} suggests that Fides didn't gain any trust for any peer in the network.
This happens because the evaluation strategy didn't have enough information at the beginning to evaluate the received data properly.
That leads to peers never gaining any trust and thus not producing any valid outputs because, with no trust, the target score and confidence ended up being $0$ as well.
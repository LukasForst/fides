\chapter{Experiments}
\label{chap:experiments}

We designed a single and comprehensive experiment that simulates a real world usage of the Fides. \todo{we need a bit more motivation here}


\section{Simulation Parameters}

\begin{itemize}
    \item number of other peers in the network, $0 <= 50?$
    \item percentage of peers in the network, that are pre-trusted (so we won't run recommendation for them and they won't be knowingly lying)
    \item for each of the peers, we will have their behavioral strategy 
    \begin{itemize}
        \item \textit{confident correct} peer - an \textbf{honest peer} that provides correct data (according to the baseline - if target is malicious, it provides TI that says it is malicious) and it is confident about its decisions
        \item \textit{uncertain} peer - an \textbf{honest peer} that can essentially provide random TI data but with low confidence in them
        \item and that
    \end{itemize}
    \item and this
\end{itemize}

\section{Sampling Threat Intelligence}
Threat intelligence, that is being shared on the peer-to-peer network and is aggregated by Fides, is generated inside Slips by various modules.
Each module provide score on its own and Slips aggregates these evaluation into a single value. 
This means, that threat intelligence is computed as a sum of independent random variables and that tends to follow normal distribution. 
For that reason, we sample threat intelligence values from the normal distribution.

As peers have different behavior, we will sample threat intelligence provided by them every time when they are asked for it.
We will characterize the peer's behavior by the threat intelligence it provides, with respect to the baseline, the ground truth of if the target is being or malicious.

As described in the previous chapters, threat intelligence consist of \textit{score} and \textit{confidence} in that score.
We use notation $\mu_{s}$ for mean threat intelligence score and $\sigma_{s}$ for the standard deviation of the score. 
Same as for score, we use $\mu_{c}$ for mean confidence and $\sigma_{c}$ for the standard deviation of the confidence. 

Fides also in some cases employs recommendation protocol, so in the simulations, the peers might be asked to provide recommendation about other peers.
They will follow their behavioral strategy, when providing the data.

Recall that a single recommendation consist of following data:
\begin{itemize}
    \item $cb_{k,j}$ - competence belief which approximates average behavior, we will be 
\end{itemize}


\begin{equation}
\begin{split}
    &\mu_{s}, \sigma_{s} \\
    &\mu_{c}, \sigma_{c}
\end{split}
\end{equation}

\section{Peer's Behavioral Patterns}
\label{sec:peers-behavioral-patterns}
For a sake of the experiments, every peer will follow one of the chosen behavior. 
We identified multiple different behavioral patterns for the being as well as for the malicious peers.

\subsubsection{Confident Correct Peer}
An honest peer that provides correct data according to the baseline. Meaning, that if the target (domain/IP address that we have threat intelligence for) is being, the peer with \textit{confident correct} behavior will provide threat intelligence that says that the target is being. 
Moreover, it will provide the data with high confidence.

\begin{equation}
\begin{split}
    \mu_{s} = 
\end{split}
\end{equation}

The very same thing applies for situation when this peer is asked to provide the recommendation for any other peer. 
The provided recommendation will reflect the real behavior of said peer and it will indicate high confidence in the recommendation.

This peer has the \textit{ideal} behavior as its data are useful and 
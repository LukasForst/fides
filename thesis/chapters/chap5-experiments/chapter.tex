\chapter{Experiments}
\label{chap:experiments}

We designed a single and comprehensive experiment that simulates a real world usage of the Fides. \todo{we need a bit more motivation here}

\section{Sampling Threat Intelligence}
\label{sec:sampling-threat-intelligence}
Threat intelligence, which is being shared on the peer-to-peer network and is aggregated by Fides, is generated inside Slips by various modules.
Each module provides a score on its own and Slips aggregates these evaluations into a single value. 
This means, that threat intelligence is computed as a sum of independent random variables and that tends to follow the normal distribution. 
For that reason, we sample threat intelligence values from the normal distribution.

As peers have different behavior, we will sample the threat intelligence provided by them every time when they are asked for it.
We will characterize the peer's behavior by the threat intelligence it provides, with respect to the baseline, and the ground truth of the target being benign or malicious.

As described in the previous chapters, threat intelligence consists of a \textit{score} and the \textit{confidence} in that score.
We use the notation $\mu_{s}$ for the mean threat intelligence score and $\sigma_{s}$ for the standard deviation of the score. 
Similarly, we use $\mu_{c}$ for mean confidence and $\sigma_{c}$ for the standard deviation of the confidence. 

Fides also in some cases employs a recommendation protocol, so in the simulations, the peers might be asked to provide recommendations about other peers.
They will follow their behavioral strategy when providing the data. 
Recall the recommendation description from section \ref{subsec:recommendations}. A single recommendation response contains $cb_{k,j}$, $ib_{k,j}$, $sh_{k,j}$, $r_{k,j}$ and $\eta_{k,j}$. 
We will be sampling those from the normal distribution as well with the corresponding pairs of $(\mu_{cb}, \sigma_{cb})$, $(\mu_{ib}, \sigma_{ib})$, $(\mu_{sh}, \sigma_{sh})$, $(\mu_{r}, \sigma_{r})$ and $(\mu_{\eta}, \sigma_{\eta})$.
Every peer will provide recommendations based on his behavioral strategy with respect to the ground truth.



\section{Peer's Behavioral Patterns}
\label{sec:peers-behavioral-patterns}
For the sake of the experiments, every peer will follow one of the chosen behaviors.
We identified multiple different behavioral patterns for the benign as well as for the malicious peers.
Every behavior is different and is defined by the $(\mu, \sigma)$ for every data we sample and by the intent the peer has in the network.
Most of the behavior depends on the baseline, which is the ground truth for any target in the system - if it is benign or malicious.
We note baseline score as $S_{B} \in \{-1, 1\}$, where $S_{B} = -1$, means the target is malicious and $S_{B} = 1$ means the target is benign.


\subsection{Confident Correct Peer}
\label{subsubsec:confident-correct-peer}
This behavior corresponds to an honest peer that provides correct data according to the baseline. 
Meaning, that if the target (domain/IP address that we have threat intelligence for) is benign, the peer with \textit{confident correct} behavior will provide threat intelligence that says that the target is benign. 
Moreover, the peer will provide the data with high confidence.

The very same thing applies to the situation when this peer is asked to provide a recommendation for any other peer. 
The provided recommendation will reflect the real behavior of said peer and it will indicate high confidence in the recommendation.
This peer has the \textit{ideal} behavior as its data are useful and correct.
The table \ref{tab:confident-correct} describes the data used for sampling the threat intelligence this peer provides.

\begin{table}[!ht]
    \centering
    \begin{tabular}{c|c|c|c}
        type & notation & $\mu$ & $\sigma$ \\
        \hline
        score & $\mu^{cc}_{s}$ & $S_{B} \cdot 0.9$ & $0.1$ \\
        confidence & $\mu^{cc}_{c}$ &  $0.9$ & $0.1$ \\
    \end{tabular}
    \caption{Confident Correct Behavior}
    \label{tab:confident-correct}
\end{table}

\subsection{Uncertain Peer}
\label{subsubsec:uncertain-peer}
This behavior simulates peers that do not have enough information to provide reasonably good data, but they are benign and honest with their behavior.
The peer can provide essentially any score but with very low confidence in said score.

\begin{table}[!ht]
    \centering
    \begin{tabular}{c|c|c|c}
        type & notation & $\mu$ & $\sigma$ \\
        \hline
        score & $\mu^{up}_{s}$ & $0.0$ & $0.8$ \\
        confidence & $\mu^{up}_{c}$ &  $0.3$ & $0.2$ \\
    \end{tabular}
    \caption{Uncertain Peer Behavior}
    \label{tab:uncertain-peer}
\end{table}

\subsection{Confident Incorrect}
\label{subsubsec:confident-incorrect-peer}
The peer with this behavior is confident about their data and the threat intelligence, but their threat intelligence is wrong.
However, this peer is still benign and is making honest mistakes.
This strategy simulates peers that were not attacked by a malicious device and they consider it benign because they do not have any information indicating malicious intent.
Thus, whenever the peer is asked to provide threat intelligence, it responds with a score that is opposite of the baseline and with a high confidence value.

\begin{table}[!ht]
    \centering
    \begin{tabular}{c|c|c|c}
        type & notation & $\mu$ & $\sigma$ \\
        \hline
        score & $\mu^{ci}_{s}$ & $-S_{B} \cdot 0.8$ & $0.2$ \\
        confidence & $\mu^{ci}_{c}$ &  $0.8$ & $0.2$ \\
    \end{tabular}
    \caption{Confident Incorrect Behavior}
    \label{tab:confident-incorrect}
\end{table}


\subsection{Malicious Peer}
\label{subsubsec:malicious-peer}
The malicious peer is going to provide wrong threat intelligence intentionally to achieve their goal of influencing the trust decisions of the local peer. 
The sampling data are the same as for the \textit{confident incorrect} (section~\ref{tab:confident-incorrect}) behavior, but the difference is that the malicious peer is providing misleading data intentionally.

This behavior simulates knowledgeable adversaries that are able to follow the Fides's protocol and their goal is to influence decisions of the local trust model.
The adversaries can be either trying to \textit{bad-mouth} or provide \textit{unfair praises}.
In our case, it does not matter why they do that, but rather the fact, that they do that intentionally and that they are providing the opposite of the baseline score with the high confidence.

We decide to design an attacker, that is trying to hide in the data and it is not providing score $\{-1, 1\}$ with the confidence of $1$ all the time, but rather uses a distribution that is close to these values.
The reason is that if the model sees that there is a peer that provides $\{-1, 1\}$ with high confidence all the time, it would be very easy to detect and penalize this behavior.

\begin{table}[!ht]
    \centering
    \begin{tabular}{c|c|c|c}
        type & notation & $\mu$ & $\sigma$ \\
        \hline
        score & $\mu^{m}_{s}$ & $-S_{B} \cdot 0.9$ & $0.1$ \\
        confidence & $\mu^{m}_{c}$ &  $0.9$ & $0.1$ \\
    \end{tabular}
    \caption{Malicious Behavior}
    \label{tab:malicious-peer}
\end{table}



\section{Environment Simulation}
\label{sec:environment-simulation}
We will be simulation a global peer-to-peer environment where the peers have one of the behaviors described in \ref{sec:peers-behavioral-patterns}.

Simulation itself have following parameters
\begin{itemize}
    \item number of peers in the network - $1..25$
    \item \% of peers that are pre-trusted and thus have higher starting service trust AND can not be malicious $0..100\%$
    \item for each type of strategy (type of peers) \% of them in the network
    \item number of targets that the local peer will be asking about and their $1..10$, labels (for example 3 peers where there are two being and one malicious)
    \item period of time when the peers will be only gaining trust (malicious peers do not lie) - $0..50$
    \item number of peers that join later during the simulation
\end{itemize}

Each peer will have following parameters
\begin{itemize}
    \item type of strategy (that includes if the peer is malicious or not
    \item is part of pre-trusted organization or not
    \item when the peer joined the network
\end{itemize}

For malicious peer 
\begin{itemize}
    \item \% of targets that the peer is going to lie about
\end{itemize}

For peers joining later in the process.
\begin{itemize}
    \item type of the joining peer (its strategy)
    \item when it appears
\end{itemize}
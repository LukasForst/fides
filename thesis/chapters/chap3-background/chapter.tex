\chapter{Background}
Here want to start with a little motivation and about ideas on top of which we built the whole thing. 
We also want to describe the terminology we're using.

\begin{itemize}
\item \textbf{Fides} - name of the trust model, performs all trust related computations
\item \textbf{service trust} - how much a trust model trust a peer to provide us a good service - it is not necessarily how much we trust the data we received from the peer as to the final computation we may include the information about peer's IP address from the Slips (what does Slips think about the IP address)
\item \textbf{target} - an IP address or domain - point of interest that can be source of network traffic and Slips has a capability to assign threat intelligence data to this object
\item \textbf{remote peer} - peer that is running somewhere on the internet and it is connected to the global peer-to-peer network
\item \textbf{local peer} - local instance of Slips that is connected to the global peer-to-peer network and runs instance of Fides
\end{itemize}

\section{Computational Model of Fides}
\label{sec:computational-model}
This section describes how does Fides come up with a single most important value $st_{i,j}$ service trust. Service trust describes how much can local peer $i$ trust remote peer $j$.
Used algorithm, that computes $st_{i,j}$, is based on SORT\cite{sort} with modifications to fit our use case - a global peer-to-peer network for sharing threat intelligence. The modifications are based on the interaction evaluation strategies proposed in \ref{subsec:interaction-evaluation} and algorithms to solve cold start problem described in \ref{subsec:cold-start-problem}.

\subsection{Intuition}
In the following pages, we describe the process top-down starting with the most important parts - service trust - and then breaking it down to bits.
There are two main ideas behind the most of the equations. 

The first one is that we want to robustly capture average behavior of the peers. In order to do that, we will be computing average behavior and standard deviations from said behavior and normalizing them.

Secondly, we will be comparing and weighting first hand experience with the remote experience. 
First hand experience is what happened between local and remote peer during time they interacted. This can be, for example, threat intelligence sharing, file sharing or the results of recommendation protocol.
Remote experience is what happened between one remote peer and another remote peer. In other words, first hand experience for peer $j$ are actions between $j$ and $z$. When $j$ shares information about these action with peer $i$, for $i$ it is a remote experience.

\begin{table}[ht]
\centering
\begin{tabular}{ c | m{20em} }
 $i$ & local peer, instance of Fides \\
 \hline
 $j$ & remote peer somewhere on the internet \\
 \hline
 $st_{i, j}$ & service trust - how much $i$ trusts $j$ that it provides good service \\
 \hline
 $r_{i, j}$ & $i$'s reputation value about $j$ \\
 \hline
 $rt_{i, j}$ & $i$'s recommendation trust about $j$ \\
 \hline
 $sh_{i, j}$ & size of $i$'s service history with $j$ \\
 \hline
 $s^{k}_{i, j}$ & $i$'s satisfaction value with interaction with peer $j$ in window $k$\\
 \hline
 $w^{k}_{i, j}$ & weight of $i$'s interaction with $j$ in $k$ \\
 \hline
 $f^{k}_{i, j}$ & fading effect of $i$'s interaction with $j$ in $k$ \\
\end{tabular}
\caption{Fides Computational Model Notation}
\label{tab:notation-computational-model}
\end{table}

\vspace{5mm}

\noindent 
The table \ref{tab:notation-computational-model} describes the most important notation we use in the following sections.

\subsection{Service Trust}
One of the major goals of the algorithm is to compute the service trust $st_{i,j}$.
We do that by weighting local experience with peer's $j$ service, with the reputation, $j$ got when it connected to the $i$. The weight here is size of the service interaction history $sh_{i,j}$ to maximal history size $sh_{max}$.

\begin{equation}\label{eq:service-trust}
st_{i,j}=\frac{sh_{i,j}}{sh_{max}} \cdot \left(cb_{i,j} - \frac{1}{2} \cdot ib_{i,j} \right) +\left(1-\frac{sh_{i,j}}{sh_{max}}\right) \cdot r_{i,j}
\end{equation}

The equation implies that the more interaction there was, between $i$ and $j$, the bigger impact on $st_{i,j}$ it has. 
In other words, the more $i$ and $j$ interact the less $i$ rely on the reputation that $i$ computed from the values provided by the network, at the time when $j$ was seen for the first time.

First part of the equation contains $cb_{i,j}$ - \textit{competence belief} - and $ib_{i,j}$ - \textit{integrity belief}. \textit{Competence belief} represents how well $j$ satisfied the $i$ with the past interactions. We measure it as an average of interactions from the past.
\textit{Integrity belief} is a level of confidence in predictability of future interactions. $ib_{i,j}$ is then measured as deviation from the average behavior $cb_{i,j}$. \cite{sort}

In order to mitigate cold start problem outlined in \ref{subsec:cold-start-problem} and in cases when there are no or little interactions between $i$ and $j$, the algorithm relies on $r_{i,j}$ - reputation value.

\subsection{Interaction Evaluations}
\label{subsec:interaction-evaluation}
In order to determine what remote peers are providing valuable data and what peers are are not, the local peer needs to be able to evaluate each interaction it had with the remote peer.
In general, there are two options how to approach this - either by designing evaluation that is protocol aware (meaning, that it understands the protocol and the data that are two peers sharing with each other), or by having an evaluation function that does not need to understand the protocol and can be used for any data.

We choose to implement both approaches and they are described in the following sections. In order to evaluate which strategy is better in what scenarios, we designed and run many simulations - their results are described in section \ref{sec:simulations-and-evaluations}.

% TODO: maybe do not use S and C when referring to score and confidence as we use "s" for satisfaction
We will use notation from table (\ref{table:interaction-eval})  when referring to peers and their interactions.
\begin{table}[h!]
\centering
\begin{tabular}{ c | m{20em} }
 $i$ & local peer \\
 \hline
 $j$ & remote peer \\
 \hline
 $T$ & target of network intelligence, domain or IP address \\
 \hline
 $k$ & evaluation window \\
 \hline
 $s^{k}_{i, j}$ & $i$'s satisfaction value with interaction with peer $j$ in window $k$\\
 \hline
 $S^{k}_{j, T}$ & score computed by the peer $j$ about target $T$ in window $k$ \\
 \hline
 $C^{k}_{j, T}$ & confidence, how much is the score correct \\
 \hline
 $S^{k}_{T}$ & aggregated score from all threat intelligence reports in window $w$ for target $T$ \\
 \hline
 $C^{k}_{T}$ & aggregated confidence
\end{tabular}
\caption{Interactions Symbols}
\label{table:interaction-eval}
\end{table}


\subsubsection{Evaluate all interactions with the same value}
\label{subsubsec:same-eval-for-all-interactions}
The strategy, that does not need to understand underlying data, their semantics nor their structure.
It is a naive approach when the trust model uses the same satisfaction value for all data it received. It does not check, if the data make sense (for example when all other peers but one are reporting that the IP address is malicious) and assigns all peers same satisfaction value $s^{k}_{i, j}$. 
The idea behind this algorithm is that when the peers are interacting for a longer time or have more interactions, they're more trustworthy.

This approach is for example used by the botnet \textbf{Sality} or by the \textbf{Dovecot} trust model. Fides implements it as $EvenTIEvaluation$ strategy with configurable satisfaction value and administrator can use this strategy if they see it as the most optimal.

The disadvantage of this approach is, that we do not penalize remote peers when they provide wrong data, because the evaluation method does not care nor understand the underlying data.
Because of that and in a case when the adversary gains the service trust of the model by following the protocol for longer time, it may significantly influence the aggregated score as the adversary has higher trust then other remote peers. If this happens, there is no way to automatically downgrade adversary's service trust.

\subsubsection{Use aggregated network intelligence for evaluation}
\label{subsub:distance-based-eval}
Because Fides is designed for the sharing and aggregating threat intelligence and understands the protocol that is being used, we can utilize this and penalize peers that are providing local peer with incorrect data.
The interaction evaluation is performed at the end of the threat intelligence sharing process, at that point, Fides already aggregated data and decided what is the aggregated network score and confidence. 
Thus, we can utilize aggregated values and use it as a base line. Then we compare it against every each remote peer's threat intelligence we received. This evaluation strategy is implemented in the Fides a as a $DistanceBasedTIEvaluation$.

% TODO: here we need to set correct indexes, because k-th interaction between local and peer i is not necessarily the same number as for the peer i+1 -> that means that for S_a we will need another number
% TODO: we need to move this to another section and to describe what is score and what is confidence
Suppose, that remote peer $j$ provided data about target $T$ to local peer $i$ in window $k$. Provided data consist of score and confidence - ($S^{k}_{j, T}$, $C^{k}_{j, T}$). Where score,  $-1 \leq S^{k}_{j, T} \leq 1$, indicates if the target is malicious ($-1$) or begin ($1$). The confidence $0 \leq C^{k}_{j, T} \leq 1$ on the other hand indicates, how much is the peer sure about its assessment of $S^{k}_{j, T}$.

In order to evaluate interaction between the local peer $i$ and remote peer $j$ we need to compute satisfaction value $s^{k}_{i, j}$. 
It holds that  $0 \leq s^{k}_{i, j} \leq 1$ - where $1$ means peer $i$ was satisfied with the interaction.
% TODO: [?] this equation can be more robust, we can for example, compute standard deviation from the S^{k}_{T} and then compare each S to nearest second quantile -> that way the equation is more robust
\begin{equation}
s^{k}_{i, j} = \left(1 - \frac{|{S}^{k}_{T} - S^{k}_{j, T}|}{2} \cdot C^{k}_{j, T}\right) \cdot C^{k}_{T}
\end{equation}

Where $S^{k}_{T}$ is final score aggregated across the reports from the peers, $C^{k}_{T}$ is aggregated confidence.

The problem in this evaluation algorithm are the situations when the aggregated confidence $C^{k}_{T}$ is close to $0$. In this case the algorithm will penalize all peers for providing any threat intelligence as the final $s^{k}_{i, j}$ is close to $0$. Another issue with this approach is that when a single honest peer has a unique information about an IP address or domain, which is significantly different then what other peers have, it is automatically penalized for not sharing the same opinion as the other peers. However, if the peer is trusted enough, it has higher impact on the aggregated value and it is not penalized too much.

\subsubsection{Use network intelligence only if the confidence is high enough}
\label{subsubsec:network-intelligence-conf-high-enough}
In order to compensate for the low confidence $C^{k}_{T}$ and penalizing all peers in algorithm explained in \ref{subsub:distance-based-eval}, this evaluation strategy considers $C^{k}_{T}$ value and employs  $DistanceBasedTIEvaluation$ only when the  $C^{k}_{T}$ is \textit{"high enough"}. In this case \textit{"high enough"} means higher then configured value by the Slips administrator - ${CT}$.
In a case when  $C^{k}_{T} < {CT}$, the algorithm fall backs to using $EvenTIEvaluation$, because it is not possible to distinguish between \textit{"good"} and \textit{"bad"} network intelligence due to low confidence of the decision. 
This strategy is implemented in Fides under the name $ThresholdTIEvaluation$.

% TODO: I'm not sure if we really need this schema
\begin{algorithm}
\caption{$ThresholdTIEvaluation$}\label{alg:threshold-ti-evaluation}
\begin{algorithmic}[1]
\State ${CT} \gets configuration$ \Comment{configuration provided by the administrator}
\If{$C^{k}_{T} < {CT}$}
	\State $s^{k}_{i, j} \gets EvenTIEvaluation()$
\Else
    \State $s^{k}_{i, j} \gets DistanceBasedTIEvaluation()$
\EndIf
\end{algorithmic}
\end{algorithm}

\noindent What should be the correct value for $CT$ from configuration is subject to evaluation in the simulations in section \ref{sec:simulations-and-evaluations}.

\subsubsection{Use local threat intelligence to evaluate network intelligence}
\label{subsub:use-local-threat-to-evaluate}
This approach uses similar equation for computing satisfaction value outlined in \ref{subsub:distance-based-eval}. However, the input is different - instead of comparing remote peer's ($j$ ) threat intelligence ($S^{k}_{j, T}$, $C^{k}_{j, T}$) to aggregated intelligence ($S^{k}_{T}$, $C^{k}_{T}$), we compare it to the threat intelligence of the local ($i$) Slips instance - ($S^{k}_{i, T}$, $C^{k}_{i, T}$). Thus the evaluation is following:

\begin{equation}
s^{k}_{i, j} = \left(1 - \frac{|{S}^{k}_{i, T} - S^{k}_{j, T}|}{2} \cdot C^{k}_{j, T}\right) \cdot C^{k}_{i, T}
\end{equation}

This approach is useful when local peer has enough information about the target, but it wants to verify the behavior of the remote peers.
To determine whether they are sending data that are somewhat correct. This strategy is implemented in Fides with name $LocalCompareTIEvaluation$.

\subsubsection{Weight local opinion with aggregated one}
Another implemented strategy combines \ref{subsub:distance-based-eval} and \ref{subsub:use-local-threat-to-evaluate} and mixes them using weight $w$, provided from the configuration.
This is good approach when the the Slips or the network has a lot of data on the target. It evaluates interactions with what local instance thinks and what the network opinion is.
What the correct mixture is is subject to simulations and configuration from the administrator.

\begin{equation}
\begin{split}
    s^{k}_{i, j} = w &\cdot \left(1 - \frac{|{S}^{k}_{i, T} - S^{k}_{j, T}|}{2} \cdot C^{k}_{j, T}\right) \cdot C^{k}_{i, T} + \\
    (1-w) &\cdot \left(1 - \frac{|{S}^{k}_{T} - S^{k}_{j, T}|}{2} \cdot C^{k}_{j, T}\right) \cdot C^{k}_{T}
\end{split}
\end{equation}

\noindent
In Fides implemented as the $WeightedDistanceToLocalTIEvaluation$.

\subsubsection{Utilize all available data in evaluation}
The goal of this strategy is to evaluate the received data with as much confidence as possible while having fully automatic process without the administrator's configuration.
In order to do that, we combine all previous strategies into one, where we utilize all available information into a single $s^{k}_{i, j}$ value.

We introduce new variables here - $p_{0}, p_{1}, p_{2}$ - which are essentially weights of the particular strategies. These weights are based on the confidence, that the strategy has in its own decision.
Note, that there is a hierarchy and order matters. 
In our case we decided to prefer decisions coming from strategy \ref{subsub:distance-based-eval}, then we add data from the \ref{subsub:use-local-threat-to-evaluate} and if the final decision still does not have confidence of $1$, we add static value configured by the administrator (noted as $S_{C}$). 
The last part - $S_{C}$ - simulates static strategy described in \ref{subsubsec:same-eval-for-all-interactions} and is set by the Slips administrator.
% TODO should we use min or the detailed version?

\begin{equation}
\label{equation:strategies-weights}
\begin{split}
    p_{0} &= {C}^{k}_{T} \\
    p_{1} &= \frac{1 - {C}^{k}_{T} + {C}^{k}_{i, T} - |1 - {C}^{k}_{T} - {C}^{k}_{i, T}|}{2} \\
    p_{1} &= min(1 - {C}^{k}_{T}, {C}^{k}_{i, T}) \\
    p_{2} &= 1 - p_{0} - p_{1}
\end{split}
\end{equation}
The weights $p_{0}, p_{1}, p_{2}$ in \ref{equation:strategies-weights}, are designed to gather as much confidence as possible. $p_{0}$ is confidence of the aggregated network data, essentially saying how much is the network sure about the given score. 
$p_{1}$ is the confidence coming from the local IPS and the $p_{2}$ is the remaining confidence to $1$.

When we have the weights, we can compute final $s^{k}_{i, j}$ where we use strategies - $p_{0} \cdot \ref{subsub:distance-based-eval}$, $p_{1} \cdot \ref{subsub:use-local-threat-to-evaluate}$ and $p_{2} \cdot \ref{subsubsec:same-eval-for-all-interactions}$. 
\begin{equation}
\begin{split}
    s^{k}_{i, j} &= \\
    &p_{0} \cdot \left[\left(1 - \frac{|{S}^{k}_{T} - S^{k}_{j, T}|}{2} \cdot C^{k}_{j, T}\right) \cdot C^{k}_{T}\right] + \\
    &p_{1} \cdot \left[\left(1 - \frac{|{S}^{k}_{i, T} - S^{k}_{j, T}|}{2} \cdot C^{k}_{j, T}\right) \cdot C^{k}_{i, T}\right] + \\
    &p_{2} \cdot S_{C}
\end{split}
\end{equation}

\noindent
This strategy is implemented in Fides under a name $MaxConfidenceEvaluation$.


\section{Cold Start Problem}
\label{sec:cold-start-problem}
A dynamic and global environment such as a global peer-to-peer network is open to anyone since any peer can freely join and leave. Because of that, the local peer will encounter many other peers that were not seen before. Therefore, the trust model does not have any information about their reliability or how much it can trust them. 
New benign peers need to be \textit{somehow} trusted by the local peer in order to be a useful part of the network. However, the local peer also needs to be able to discover new malicious peers that are trying to gain its trust.

The problem of how to know something about a new entity in order to quickly work better is called the \textit{Cold Start Problem}~\cite{christensen2014hybrid}. For Fides it means how to compute a good initial trust for new unknown peers. 

We selected several solutions to this issue, which are all implemented in Fides. Fides also combines them according to a provided configuration with the aim to achieve the best result for the cold start problem with adversarial peers.

\subsection{Static Initial Trust}
\label{subsec:static-initial-trust}
In this approach, whenever the trust model encounters a new peer, it assigns a static value as an initial trust. The value is assigned by pre-choosing some third-party trust models in the configuration.

For example, in the \textbf{Dovecot} trust model~\cite{dita}, every peer starts with trust $1$ (highest possible), and various interactions can lower the trust in the peer to $0$. In other words, the trust model considers new peers honest from the beginning, and only during this time their reputation can be lowered when they perform incorrect interactions or are discovered as a malicious peer.

On the other hand, the \textbf{Sality} botnet~\cite{falliere2011sality} uses a value called \textit{goodcount} as a counter of good interactions with any other peer, the larger the \textit{goodcount}, the greater trust the local peer has on the remote peer. The goodcount for each new peer starts with $0$ in Sality. Meaning, that the botnet does not trust fresh peers at all and they can gain trust only by following the Sality protocol.

The model of \textit{static initial trust} is easy to implement, but it requires assumptions about the network. If the network is considered \textit{mostly benign}, it might be safe to use an initial trust of $1$, however for highly adversarial networks using an initial trust of $1$ might be dangerous and it is better to use $0$. 
Using low initial trust and no mechanism how to gain more trust fast means, that the benign peers that joined recently, don't affect the final decisions of the model even though they might have useful information about adversaries.

Static initial trust is supported by Fides as a form of fallback when no other cold start technique is used. The administrator provides a configuration that contains the initial reputation for each new peer.

\subsection{Pre-Trusted Peers}
\label{subsec:pre-trusted-peers}

This master thesis was done simultaneously to the master thesis on global P2P security TI sharing by Bc. Martin Řepa~\cite{nl} which implements the new idea of pre-trusted peers in organizations for the Slips IPS. Therefore, we worked with the concept of pre-trusted organizations which have pre-trusted peers. In this case, Fides can use this knowledge and assign a higher or lower trust to new peers.

The global P2P framework implemented by Řepa supports these type of peers and provides a cryptographically-secure way how to identify a single peer in the network, and its membership in an organization.
This allows Fides to \textit{pre-trust} specific peers or all the peers from organizations by assigning them an initial value.

Fides can be configured to use pre-trust in two different ways. First, to assign the pre-trusted peers an initial \textit{reputation}. This means that the peer will have an initial reputation, but it will be required to interact with the local peer and it will slowly change that initial reputation according to its interactions with others. All the interactions will be evaluated and Fides will compute a service trust for the peer, as described in Section~\ref{sec:interaction-evaluation-strategies}. 

Second, Fides can use the initial pre-trusted value read from the configuration as the final service trust. This effectively means that Fides will not evaluate any data received from the pre-trusted peer and this service trust will be kept forever. 

This configuration for the Fides is called \textit{enforceTrust}, if it is enabled and thus $enforceTrust = False$ is set in the configuration, Fides uses first variant where the trust for the peer will move during the interactions. When administrator uses $enforceTrust = True$, Fides uses second option and fixates the service trust for the peer to a set \textit{pre-trust}.

Both options helps to solve the cold start problem for specific peers and organizations, as they will start with a high reputation or fixed service trust. Which organization or peer to trust is completely left to the administrator of Slips. 
The inspiration for which organization or to which peer to trust provides, for example, Tor and their directory authorities~\cite{torauth}. However as the administrator needs to know the identity of the peers or organization, it does not solve the cold start problem globally for all peers.

\subsection{Recommendations}
\label{subsec:recommendations}
As the local peer might have multiple remote peers that it trusts enough, Fides uses these relationships to ask the remote peers about how much they trust a new peer. Fides only asks for recommendations when the local peer finds a new peer.  

The recommendation system introduces new attack vectors, that can be exploited by adversaries either by getting trust for the malicious peer or by lowering trust in honest peers that might have some threat intelligence about the malicious actor. 
These attacks are called \textit{bad-mouthing} and \textit{unfair praises} and we need to consider them and implement countermeasures.

Because of the possible attacks, the local peer should not solely rely on the network recommendations when computing the final service trust for the fresh peer. In case, when the recommending peers are malicious, it might skew the decisions of the local peer for the time being.
In order to solve this, when computing the final service trust for the remote peer, the local peer should take into account its own interaction with the peer as well as the received recommendations.

Moreover, the local peer should request recommendations only if it has \textit{enough} trusted remote peers, otherwise, it can expose itself to \textit{bad-mouthing} and \textit{unfair praises} attacks more easily.

\vspace{7mm}

Fides employs recommendation systems based on SORT~\cite{sort} and combines it with the pre-trusted peers~(\ref{subsec:pre-trusted-peers}) as well as with the static initial trust~(\ref{subsec:static-initial-trust}) as a fallback when no other option is available due to constraints such as having not enough trusted peers.
The algorithm used for the recommendation system is explained in detail in the section~\ref{sec:computational-model}.

\section{Network Intelligence Aggregation}
\label{sec:network-intelligence-aggregation}
Fides is a trust model designed for global peer-to-peer networks of Slips instances.
It is designed to support Slips in detecting malicious actors on the network and enables threat intelligence sharing between peers of Slips instances.
Because Slips was designed to be as modular as possible, Fides is effectively running as a module that provides aggregated threat intelligence to Slips. 
In other words, Fides provides a view of what the network thinks about some threat intelligence target. This is necessary so Slips can have a unique \textit{view} of the network on a specific Threat Intelligence.
Fides needs to aggregate elements of threat intelligence from remote peers into a single value that is then presented to Slips.

Fides needs to say that some reports are better than others, based on the service trust the local peer has in the remote peer (previously computed as $st^{k}_{i, j}$).
Thus Fides needs to weigh every report based on this trust and come up with an aggregated score $S^{k}_{T}$.
Apart from the aggregated score, Fides needs to compute the aggregated confidence $C^{k}_{T}$ that expresses how confident $i$ is about the aggregated score $S^{k}_{T}$ that was computed in the previous step.

Once aggregated, the computed score and confidence ($S^{k}_{T}$, $C^{k}_{T}$) are sent to Slips to report data on target $T$.
Apart from sending to Slips, these same values can be also used to evaluate the interaction of the remote peers, depending on the selected interaction evaluation strategy. We describe this more in depth in Section~\ref{sec:interaction-evaluation-strategies}.

We designed and implemented two different functions for aggregating threat intelligence and computing $S^{k}_{T}$ alongside with $C^{k}_{T}$.
Both of them are implemented in Fides under their respective names and which method performs better under what circumstances is a subject of the experiments in Chapter~\ref{ch:experiments}.

\subsection{AverageConfidenceTIAggregation}
\label{subsec:AverageConfidenceTIAggregation}

In this method, the aggregated score $S^{k}_{T}$ is the sum of $S^{k}_{j, T}$, which is the score sent by each peer $j$ about target $T$ in time window $k$; weighed with the normalized service trust that $i$ computed for peer $j$, denoted $wst^{k}_{i, j}$. The sum is done over the set of remote peers that provided a report to $i$ for $T$ in time window $k$, denoted $R^{k}_{i, T}$. We calculate it in Equation~\ref{eq:ti_aggregation_score}.

\begin{equation}
\begin{split}
    S^{k}_{T} &= \sum_{{j}\in R^{k}_{i, T}} wst^{k}_{i, j} \cdot S^{k}_{j, T}
\end{split}
\label{eq:ti_aggregation_score}
\end{equation}

\noindent
The normalized service trust $wst^{k}_{i,j}$ used as weight is computed as:

\begin{equation}
\begin{split}
    wst^{k}_{i,j} &= \frac{1}{\sum_{{j}\in R^{k}_{i, T}} st^{k}_{i, j}} \cdot st^{k}_{i, j} \\
\end{split}
\label{eq:normalized_service_trust_ti_aggregation}
\end{equation}

\noindent
Equation~\ref{eq:normalized_service_trust_ti_aggregation} estimates the percentage that the service trust on $j$ $st^{k}_{i, j}$ has relative to the total sum of service trust received by $i$ for all peers, for this target $T$, in time window $k$.

We compute the aggregated confidence $C^{k}_{T}$ for this strategy as:

\begin{equation}
\begin{split}
    C^{k}_{T} &= \frac{1}{|R^{k}_{i, T}|} \cdot \sum_{{j}\in R^{k}_{i, T}} st^{k}_{i, j} \cdot C^{k}_{j, T}
\end{split}
\end{equation}

\noindent
Which is an average over all the peers that sent to $i$ a report on $T$ in time window $k$, of the weighted confidence sent by peer $j$ on target $T$ on time window $k$. The weight is done by the service trust that $i$ has on $j$ on time window $k$.


\subsection{WeightedAverageConfidenceTIAggregation}
\label{subsec:WeightedAverageConfidenceTIAggregation}

This strategy uses Equation~\ref{eq:ti_aggregation_score} to compute the aggregated score $S^{k}_{T}$ similarly to the $AverageConfidenceTIAggregation$ in Section~\ref{subsec:AverageConfidenceTIAggregation}.
However, the way how this strategy calculates $C^{k}_{T}$ is different. 
Instead of using the service trust $st^{k}_{i, j}$ to determine the correct trust in the confidence $C^{k}_{j, T}$ submitted by peer $j$ and then diving it by the number of peers, it uses the normalized service trust $wst^{k}_{i,j}$ computed in Equation~\ref{eq:normalized_service_trust_ti_aggregation} that already contains the weight of the peers in the final decision.

\begin{equation}
\begin{split}
    C^{k}_{T} &= \cdot \sum_{{j}\in R^{k}_{i, T}} wst^{k}_{i,j} \cdot C^{k}_{j, T}
\end{split}
\end{equation}

\section{Intrusion Detection/Prevention System}
\label{sec:intrusion-detection-prevention-system}
An Intrusion Detection System (IDS) is a system that that continuously monitors a network for malicious activity and reports its findings to the network administrator.
An Intrusion Prevention System (IPS) can be seen as an extension of the IDS. 
It is a network security tool that that continuously monitors a network for malicious activity and takes action to prevent it, including reporting, blocking, or dropping it, when it does occur \cite{vmware}.
There are plenty of commercial as well as open source solutions that offer IDS/IPS capabilities, one of the is for example Slips.

\section{Slips}
\label{sec:slips}
Slips is an open source Intrusion Prevention System developed by the Stratosphere group of Faculty of Electrical Engineering at the Czech Technical University in Prague.
Slips uses machine learning to detect malicious behaviors in the network traffic and it was designed to focus on targeted attacks, to detect of command and control channels, and to provide good visualization for the analyst.\cite{slips}
Its design allows developers to extend Slips by developing new modules, that add new functionality such as additional machine learning models for network analysis or enable Slips to communicate over the network with other Slips instances.

\section{Threat Intelligence}
\label{sec:threat-intelligence}
Threat intelligence is the provision of evidence-based knowledge about existing or potential threats \cite{threatintelligence}.
Such knowledge is produced by, for example, intrusion detection/prevention systems. 
In this thesis, we will operate with the threat intelligence generated by Slips.
Slips uses multiple internal detection modules, where each provide its own assessment of the traffic flow. 
Each assessment is then taken in account when Slips aggregates it and produces threat intelligence that consist of score and confidence.
Score explains how much does Slips think the IP or domain is malicious or being.
Confidence then describes to what extent does Slips believe that the score is correct.
In this thesis, whenever we refer to threat intelligence, we refer to this score and confidence.


\section{Peer-to-Peer Networks}
\label{sec:peer-to-peer-networks}
Peer-to-peer network is a distributed network of computers without direct hierarchy where nodes (peers) communicate with each other directly, without using centralized server, or any authority.\cite{schollmeier}.
Unlike more standard client-server networks, where the server is designed to provide data and client consumes them, in peer-to-peer networks all peers are even in terms of permissions and what services do they provide.
Moreover, the peers are free to join and leave the network any time, which makes the networks highly dynamic and it does not provide much guarantees about data availability.
Nowadays the peer-to-peer networks are mostly used for file sharing which was popularized by Napster in 1999\cite{saroiu}.
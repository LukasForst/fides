\section{Intrusion Detection/Prevention System}
\label{sec:intrusion-detection-prevention-system}
An Intrusion Detection System (IDS) is a system that continuously monitors a network for malicious activity and reports its findings to the network administrator~\cite{bace2001intrusion}.

An Intrusion Prevention System (IPS) can be seen as an extension of an IDS. 
It is a network security tool that continuously monitors a network for malicious activity and takes action to prevent it, including reporting and blocking when it does occur~\cite{zhang2004intrusion}.
There are plenty of commercial and open-source solutions that offer IDS/IPS capabilities. One of them is Slips.


\section{Slips}
\label{sec:slips}
Slips is an open-source behavioral Intrusion Detection and Prevention System developed by the Stratosphere Research Laboratory of the Faculty of Electrical Engineering at the Czech Technical University in Prague.
Slips uses machine learning to detect malicious behaviors in the network traffic. It was designed to focus on targeted attacks, detect command and control channels, and provide good visualization for the analyst~\cite{slips}.
Its design allows developers to extend Slips by developing new modules that add new functionality, such as additional machine learning models for network analysis, or enable Slips to communicate over the network with other Slips instances.

As the goal of the thesis is to design and build a trust model that cooperates with the IDS, we chose Slips as our IDS/IPS mainly because it is a highly modular system.
Its modularity allowed us to integrate deeply into the system without the need for architectural changes on it.

\section{Threat Intelligence}
\label{sec:threat-intelligence}
Threat intelligence is the provision of evidence-based knowledge about existing or potential threats~\cite{threatintelligence}.
Such knowledge is produced by, for example, intrusion detection/prevention systems.
When the prevention systems receive the threat intelligence, they act according to that.
For example, if an endpoint detection system provides threat intelligence that claims that IP $x.y.z.w$ is malicious, a blocking system with access to the firewall will block access from and to this IP address.

This introduces a trust relationship between the system that detects the malicious behavior and the system that can block it.
The system taking actions, for example, blocking the IP address, needs to trust the threat intelligence that it received from other systems in the network.
This also means that the trust in some threat intelligence is an important aspect when sharing and acting according to the said threat intelligence.

In this thesis, we will operate with the threat intelligence generated by Slips.
Slips uses multiple internal detection modules, where each provides its assessment of the traffic network flows. 
Each module's assessment is then taken into account when Slips aggregates it and produces threat intelligence that consists of score and confidence.
The threat intelligence (score and confidence) refers to a target - primarily to an IP, domain, hashes of files, or any other unique identification of the subject that Slips classified with score and confidence.

The score is a threat intelligence indicator and explains how much Slips thinks the target is malicious or benign.
Confidence then describes to what extent Slips believe that the score is correct.
In this thesis, whenever we refer to threat intelligence, we refer to this score and confidence provided by Slips.

\section{Peer-to-Peer Networks}
\label{sec:peer-to-peer-networks}
A peer-to-peer network is a distributed network of computers without a direct hierarchy where nodes (peers) communicate with each other directly, without using a centralized server or any authority~\cite{schollmeier}.
Unlike more traditional client-server networks, where the server provides data and the client consumes them, in peer-to-peer networks, all peers are usually similar in terms of permissions and what services they provide.
Peers are free to join and leave the network at any time, making the network highly dynamic, and it does not provide any guarantees about data availability.
Nowadays, peer-to-peer networks are mostly used for file sharing, which Napster popularized in 1999~\cite{saroiu}.
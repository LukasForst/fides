\chapter{Conclusion}
\label{ch:conclusion}

In this thesis, we proposed, designed, and implemented a trust model for distributed multi-agent environments of intrusion prevention systems \textbf{Fides}.
Fides enables threat intelligence sharing in global peer-to-peer networks of Slips agents and allows them to cooperate against malicious actors without compromising the security of the local environment. 
Moreover, even though we fine-tuned the model for threat intelligence sharing, it is generic and can be easily modified for sharing any other IoC data.

Fides can recognize that the peer belongs to different organizations. If the administrator trusts them, Fides can utilize this knowledge to provide better results and be more confident in its decisions.
Thanks to its granular configuration, Fides allows the administrator to set up intelligence sharing between specific organizations or share specific data only with highly trusted peers.
This ensures that Fides shares only data that can be shared and does not leak any sensitive information about the infrastructure it runs in.

The trust model itself is generic and can be integrated into any modular intrusion detection or prevention system. We chose to implement an integration with Slips~\cite{slips}, and as a part of our work, we developed a Slips module that uses Fides for sharing and aggregating threat intelligence.
The whole trust model and the Slips module are open-source and available as free software~\cite{fidesGithub}.

In this thesis, we explore topics related to intrusion prevention systems~(\ref{sec:intrusion-detection-prevention-system}), threat intelligence~(\ref{sec:threat-intelligence}), and peer-to-peer networks~(\ref{sec:peer-to-peer-networks}) alongside with the state of the art in managing trust relationships with trust models~(\ref{sec:trust-in-p2p}) in Chapter~\ref{ch:previous-work-background}.
We conclude that we need a solution that is flexible enough to allow us to share sensitive threat intelligence in highly adversarial peer-to-peer networks.

In the following Chapter~\ref{ch:trust-model-design} we propose the trust model Fides and provide a high-level overview of its life cycle and the foundation of its calculations~(\ref{sec:general-overview-of-fides}).
The following sections discuss the problems that trust models face in peer-to-peer networks in general~(\ref{sec:cold-start-problem}).
After that, we analyze common attack vectors on the trust models~(\ref{sec:attack-vectors}) and outline the taxonomy of possible attacks on Fides~(\ref{sec:taxonomy-of-attacks}).
Once we outline all necessary preconditions, we deep dive into a computational model of Fides~(\ref{sec:computational-model}), including multiple options for evaluating interactions between the peers~(\ref{sec:interaction-evaluation-strategies}).
Because the trust model operates with threat intelligence, we propose multiple ways to aggregate network threat intelligence into a single value consumed by Slips~(\ref{sec:network-intelligence-aggregation}).

The architecture and implementation details of the trust model are then described in Chapter~\ref{ch:architecture} where we explain how Fides communicates in the network, how it can verify the data it receives, and how it ensures that no privacy-sensitive data are sent to the network.

In the next Chapter~\ref{ch:experiments}, we designed extensive test scenarios~(\ref{sec:environment-simulation}), explained how we evaluate each simulation~(\ref{sec:experiments-evaluation}), and then created and implemented an evaluation framework~(\ref{sec:simulations-execution}) that allowed us to simulate any possible setup in any possible environment.

We then evaluated the results of the experiments and the trust model in the Chapter~\ref{ch:results}, where we discovered that with a particular setup and under a condition that Fides talks to at least 25\% of pre-trusted peers, it could eventually identify targets correctly no matter how the rest of the peers behave.

Results have shown that Fides is a viable solution that tackles the problem of trust relationships in sensitive environments for threat intelligence sharing. 
We proved that Fides performs well and thus can be used to extend the existing implementation of Slips as a new module for sharing threat intelligence in the global peer-to-peer network.
Thanks to all the configurations and optimizations that Fides brings for usage inside the organizations, we believe that Fides enables the broader adoption of Slips Intrusion Prevention System across the broader spectrum of organizations.
We believe that Slips, combined with Fides, will enable more possibilities for sharing the threat intelligence over the internet and improve the overall security of the networks that use Slips for the defense against intruders.

However, we believe that some areas and characteristics can be improved. We write about them more in the next Section~\ref{sec:future-work}.

\section{Future Work}
\label{sec:future-work}
Even though we achieved our goal of designing a resilient trust model for sharing threat intelligence, there are multiple areas where the model can improve either by exploring different approaches or by implementing new additions to the data flow.

Most importantly, it is clear from the evaluation of simulations in Chapter~\ref{ch:results}, that Fides performs differently with different setups under different conditions.
Even though we are able to manually pick settings that ensure the best performance, this is not a perfect solution for real-world scenarios.
For that reason, we propose to explore further the following approaches.

\subsection{Exploring Interaction Evaluation Strategies}
\label{subsec:exploring-interaction-eval-strategies}
The trust model developed in this thesis is generic and it is heavily relying on the interaction evaluation function.
This means that the performance of the trust model is as good as the evaluation function.

In this work, we explored evaluation methods that are using only data from a single time window to evaluate the interactions.
However, the local peer might store the complete interaction history with all other peers and whenever it finds out that some peer reported threat intelligence, that proved to be correct after some time, the reporter's service trust should benefit from such discovery.
Another way might be storing the whole interaction history and using machine learning techniques, to discover irregularities in provided data during all communication windows.

In other words, all interaction evaluation strategies described in Section~\ref{sec:interaction-evaluation-strategies} do not utilize the knowledge from the past, or the history of the interactions.
We believe that this is an interesting space to explore more as it might lead to better performance of Fides in real-world scenarios.

\subsection{Exploring Threat Intelligence Aggregation Methods}
\label{subsec:exploring-threat-intelligence-aggregation-methods}
No threat intelligence aggregation methods explored in this thesis, described in the Section~\ref{sec:threat-intelligence}, use any other information than ones provided by the network at a single time window.
By incorporating information from the recent history, the aggregation might improve the overall detection performed by the trust model.

Moreover, there might be better ways how to aggregate the threat intelligence, or maybe the combination of multiple approaches to one might improve the final confidence in trust model decisions.
As we saw in Chapter~\ref{ch:results}, the combination of interaction evaluation and threat intelligence aggregation influences the performance of the trust model to great extent.

\subsection{Possible Mitigation of Sybil Attack}
\label{subsec:possible-mittigation-of-sybil-attack}
When analyzing the possible attack vectors for Fides, we described the Sybil attack in Section~\ref{subsec:whitewashing-and-sybil-attack} and we stated that Fides is vulnerable to this type of attack.
However, results have shown that this is true only in cases when the attacker owns more than 75\% of the network. 
If the attacker controls 75\% of the network or less, Fides has a way how to defend itself and make the correct decisions about the threat intelligence.

There are essentially two possible ways how to mitigate this type of attack even for cases where the attacker controls more than 75\% of the network.

One option is to introduce restrictions to Fides, where it uses data at most from the 75\% of peers that are not pre-trusted, and the rest, 25\% comes from the pre-trusted peers and organizations. All other data from other peers in the network would not be considered during the threat intelligence aggregation.
That would ensure that Fides always only considers a limited amount of data so it will not be vulnerable to an attacker who controls more than 75\% of the network.

The second possible solution would be making it \textit{computationally hard} for new peers to join the peer-to-peer network or generate a new peer identity.
So the attacker would need to spend either time or computational resources when generating new peer identities.
This directly does not prevent a malicious actor to perform this type of attack, but it makes it significantly harder.
However, this mitigation would need to be implemented in Iris instead of Fides.

\subsection{Adding Threat Intelligence Challenges}
\label{subsec:adding-threat-intelligence-challenges}
Fung et al~\cite{fung2008trust} explore an interesting idea of creating initial trust in remote peers by giving them \textit{computational challenges}.
In our case, challenges might be asking the remote peer for threat intelligence about the target, that the local Slips know very well and with high confidence.
When the trust model receives a response it uses the interaction evaluation strategy described in Section~\ref{subsec:use-local-threat-to-evaluate} to evaluate the received data.
The trust model can then ask multiple times to have more interactions and thus effectively \textit{probing} the remote peer and getting an estimate about its future behavior.

By using this approach, one can either replace the recommendation system or greatly improve it in cases when the trust model does not have enough pre-trusted peers.
The disadvantage might be a higher amount of messages sent across the network when the new peer is registered by the network, which could eventually lead even to a denial of service attack on the newcomers. 

This method should be explored more in detail and as the Fides is quite flexible, it can be easily incorporated into the trust model as well.

\subsection{Threat Intelligence Sharing Motivation}
\label{subsec:threat-intelligence-sharing-motivation}
The designed peer-to-peer network for threat intelligence sharing unfortunately does not promote data sharing at all.
The peers sharing threat intelligence are not gaining any benefit other than gaining trust in the eyes of other peers.
As the gained trust brings little to no benefit by itself, there is a lack of incentive for the peers to share their threat intelligence with the network.

However, we can see data and threat intelligence filtering as one of the ways how to motivate other peers to share their threat intelligence.
As we described in Section~\ref{subsec:data-filtering}, the Fides administrator can choose what threat intelligence is shared with which peers by configuring confidentiality levels with required service trust.
Using this approach, the threat intelligence is shared only with high trusted peers.
That would result in an incentive for the peers to gain the service trust which they do by sharing their own threat intelligence with the network.

Another approach is exploring the idea of \textit{payments} for the threat intelligence where the peers use some sort of cryptocurrency to \textit{pay} for the received threat intelligence and they receive payments for their own threat intelligence they shared.
This would give the peers an incentive to share the threat intelligence and ask for it only when they need it.
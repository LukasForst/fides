\chapter{Architecture}
\label{ch:architecture}
Slips is a modular software. Each module is designed to perform a specific detection in the network traffic~\cite{slips}.
The modules can also be used to extend Slips with any additional functionality directly. 
Fides was designed for seamless interoperability with Slips, and in addition to the generic trust model, we developed the Fides module for Slips.
In this chapter, we describe the architecture of Fides and how it interacts with the network and with Slips.

\begin{figure}[ht]
    \centering
    \includegraphics[width=0.4\textwidth]{assets/redis_channels.jpeg}
    \caption{Fides high-level architecture that visualizes communication between Fides, Iris and Slips.}
    \label{fig:high-level-architecture}
\end{figure}

From the high-level perspective (see Figures~\ref{fig:high-level-architecture} and \ref{fig:fides-api-network}), the trust model Fides communicates with two different systems - Slips~\cite{slips} and the network layer Iris~\cite{nl}.
Fides manages the trust relationships in the network, aggregates threat intelligence data, and communicates with Slips. The communication with the remote peers in the network is facilitated by Iris. 
Slips then produces and consumes the threat intelligence and defends the network against intruders.

Fides exposes and consumes an API~\cite{api} built using the Redis channels for both parts (Figure~\ref{fig:high-level-architecture}).
The messages and API calls are consumed using the JSON~\cite{json} data format.

Redis is an in-memory data structure store that supports asynchronous channels and a publish-subscribe model~\cite{redis}. Moreover, it can also persist data on disk if required.
We chose to employ Redis channels as the medium that allows communication between the Iris and Fides and allows them to use their respective APIs because Slips already uses Redis for its internal communication between modules. It brings no additional overhead to run Fides with its network layer.

\section{Fides \& Network Access}
\label{sec:fides-and-network-access}

\begin{figure}[ht]
    \centering
    \includegraphics[width=1.0\textwidth]{assets/communication_architecture.jpeg}
    \caption{Communication between Fides, Iris and Slips}
    \label{fig:fides-api-network}
\end{figure}

Fides itself is a trust model and it does not interact with the network directly, but rather it exposes an API that can be used either to receive the information from the network or for sending the requests back to the network.
Thanks to this design, where all business logic is separated from the network layer, Fides is highly modular and it does not depend on the network layer implementation.
The network layer Iris then performs all data transfers and facilitates all communications with the remote peers.
It also facilitates finding new peers and ensuring that all requests from Fides are dispatched to the correct recipients.
In the eyes of Fides, the network layer is a \textit{black box} and it does not need to know, how is the network layer implemented.
See figure~\ref{fig:fides-api-network} for a high-level overview of the communication.

The network layer, \textbf{Iris}, was developed by Bc. Martin Řepa in~\cite{nl} where Řepa describes how Iris works in detail and what protocols are used to safely deliver necessary information and messages between the instances of Fides.

\section{Implementation}
\label{sec:implementation}
Because Fides was designed to integrate with Slips, we were constrained by the Slips implementation~\cite{slips} and thus Fides is implemented in Python~\cite{python} in version 3.8.

The code is versioned by Git~\cite{git} and published on GitHub~\cite{github} in the repository \href{https://github.com/stratosphereips/fides}{github.com/stratosphereips/fides}~\cite{fidesGithub}.
We choose to use Conda~\cite{conda} for managing the dependencies and Python versions.

\subsection{Configuration}
\label{subsec:configuration}
Our trust model contains many different configuration options either related to the computational model itself or to the data persistence or identity of the local peer.

Computational model settings are for example the threat intelligence aggregation methods described in section~\ref{sec:network-intelligence-aggregation} or the interaction evaluation strategy from section~\ref{sec:interaction-evaluation-strategies}.
As the trust model needs only a single method, the administrator needs to define which one of these functions should be used.

The configuration itself is in a single YAML~\cite{yaml} file that is in the repository root in \href{https://github.com/LukasForst/fides/blob/master/fides.conf.yml}{\textit{fides.conf.yml}}~\cite{fidesGithub}.
This file is loaded and validated during the trust model startup and is used to provide all possible configuration options for the trust model.

\subsection{Persistence}
\label{subsec:persistence}
Fides stores trust-related data such as past interactions, cached network opinions, service trust, recommendations, etc. inside the database.
The database layer was implemented as an abstract part and can be easily replaced in the future.
As of now, we have two different implementations. An in-memory database and a database that stores data in Redis.
However, thanks to its modularity, different persistence solutions can be easily implemented.

\subsection{Data Filtering}
\label{subsec:data-filtering}
Part of the configuration is the section about data confidentiality and sharing of the threat intelligence with other peers.
Fides allows operators to choose what threat intelligence will be shared, when, and to whom.

For example, if the threat intelligence received from the local Slips instance contains a \textit{confidentiality level}, the operator can enforce that only peers with \textit{high} service trust will receive this threat intelligence when they ask for it.

The confidentiality level, $cl$, $0 \leq cl \leq 1$, defines how sensitive or confidential the threat intelligence is where $cl = 0$ means public information that can be shared with anybody and $cl = 1$ secret information that should not be shared at all.

The Fides administrator can then specify what service trust $st$ is required for what confidentiality level, $cl$, in the configuration~(section \ref{subsec:configuration}) .
If this configuration is in place, whenever a remote peer ($j$) asks for threat intelligence and the local ($i$) Slips has the requested threat intelligence, Fides verifies that $st_{i, j} \geq cl$ before providing the intelligence to the remote peer.

This mechanism ensures that Slips does not leak information that is private or somehow more sensitive than the others.